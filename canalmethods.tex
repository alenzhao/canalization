\section*{Methods}

\subsection*{Transmission model}

To characterize the joint epidemiological, genealogical, antigenic and spatial patterns of influenza, we implemented a large-scale individual-based model.  This model consists of daily time steps, in which the states of hosts and viruses are updated.  Hosts may be born, may die, may contact other hosts allowing viral transmission, or may recover from infection.  Viruses may mutate in antigenic phenotype.  Each simulation ran for 40 years of model time.  

Hosts in this model are divided between three regions: North, South and Tropics.  There are 30 million hosts within each of the three regions, giving $N = 9 \times 10^{7}$ hosts.  Host population size remains fixed at this number, but vital dynamics cause births and deaths of hosts at a rate of $1 / 30$ years $= 9.1 \times 10^{-5}$ per host per day.  Within each region, transmission proceeds through mass-action with contacts between hosts occurring at an average rate of $\beta = 0.36$ per host per day.  Regional transmission rates in temperate regions vary according to sinusoidal seasonal forcing with amplitude $\epsilon = 0.15$ and opposite phase in the North and in the South.  Transmission rate does not vary over time in the Tropics.  Transmission between region $i$ and region $j$ occurs at rate $m\,\beta_i$, where $m=0.001$ and is the same between each pair of regions and $\beta_i$ is the within-region contact rate.   Hosts recover from infection at rate $\nu = 0.2$ per host per day, so that $R_0$ in a naive host population is 1.8.  There is no super-infection in the model.

Each virus possesses an antigenic phenotype, represented as a location in Euclidean space.  Here, we primarily use a two-dimensional antigenic location.  After recovery, a host `remembers' the phenotype of its infecting virus as part of its immune history.  When a contact event occurs and a virus attempts to infect a host, the Euclidean distance from infecting phenotype $\phi_v$ is calculated to each of the phenotypes in the host immune history $\phi_{h_1}, \dots, \phi_{h_n}$.  Here, one unit of antigenic distance is designed to correspond to a twofold dilution of antiserum in a hemagglutination inhibition (HI) assay \cite{Smith04}. The probability that infection occurs after exposure is proportional to the distance $d$ to the closest phenotype in the host immune history.  Risk of infection follows the form $\rho = \textrm{max}\{d\,s,1\}$, where $s=0.07$.  Cross-immunity $\sigma$ equals $1-\rho$.  The initial host population begins with enough immunity to slow down the initial virus upswing and place the dynamics closer to their equilibrium state; initial $R$ was 1.28.

Our model follows Gog and Grenfell \cite{Gog02} in representing antigenic distance as distance between points in a geometric space.  By forcing one-dimension to directly modulate $\beta$, Gog and Grenfell find an orderly replacement of strains.  Kryazhimskiy et al. \cite{Kryazhimskiy07} use a two-dimensional strain-space, but enforce a cross-immunity kernel that directly favors moving along a diagonal line away from previous strains.  Our model does not `build in' the one-dimensional direction of antigenic drift, which instead emerges dynamically from competition among strains.

The initial virus population consisted of 10 infections each with the identical antigenic phenotype of $\{0,0\}$.  Over time viruses evolve in antigenic phenotype.  Each day there is a chance $\mu = 10^{-4}$ that an infection mutates to a new phenotype.  This mutation rate represents a phenotypic rate, rather than genetic mutation rate, and can be thought of as arising from multiple genetic sources.  When a mutation occurs, the virus's phenotype is moved in a completely random direction $\sim \textrm{Uniform}(0,360)$ degrees. Mutation size is sampled from the distribution $\sim \textrm{Gamma}(\alpha,\beta)$, where $\alpha$ and $\beta$ are chosen to give a mean mutation size of 0.6 units and a standard deviation of 0.4 units.  This distribution is parameterized so that mutation usually has little effect on antigenic phenotype, but occasionally has a large effect.  This is similar to the neutral networks implemented by Koelle et al. \cite{Koelle06}, wherein most amino acid changes result in little decrease to cross-immunity between strains, but some changes result in large jumps in cross-immunity.

\subsection*{Model output}

Daily incidence and prevalence are recorded for each region.  During the course of the simulation, samples of current infections are taken from the evolving virus population at a rate proportional to prevalence.  Each viral infection is assigned a unique ID, and in addition, infections have their phenotypes, locations and dates of infection recorded.  In this model, viruses lack sequences and so standard phylogenetic inference of the evolutionary relationships among strains is impossible.  Instead, the viral genealogy is directly recorded.  This is made possible by tracking transmission events connecting infections during the simulation; infections record the ID of their `parent' infection.  Proceeding from a sample of infections, their genealogical history can be reconstructed by following consecutive links to parental infections.  During this procedure, lineages coalesce to the ancestral lineages shared by the sampled infections, eventually arriving at the initial infection introduced at the beginning of the simulation.  Commonly, phylodynamic simulations generate sequences that are subsequently analyzed with phylogenetic software to produce an estimated genealogy \cite{Ferguson03,Koelle06,Koelle10}.  This step of phylogenetic inference is imperfect and computationally intensive, and by side-stepping phylogenetic reconstruction we arrive at genealogies quickly and accurately.  Other authors have implemented similar tracking of infection trees \cite{Volz09,Odea11}.  This genealogy-centric approach makes many otherwise difficult calculations transparent, such as calculating lineage-specific region-specific migration rates (\ref{spatial}) and lineage-specific mutation effects (\ref{mutspectrum}).

Infections are sampled at a rate designed to give approximately 6000 samples over the course of the simulation, with genealogies constructed from a subsample of approximately 300 samples.  The results presented in \ref{incmaptree}--\ref{immunity} represent a single representative model output; one hundred replicate simulations were conducted to arrive at statistical estimates. 

%%% REFERENCES %%%
\bibliographystyle{plos2009}
\bibliography{/Users/bedfordt/Documents/bedford}
